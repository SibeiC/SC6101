\documentclass[conference,nofonttune]{IEEEtran}
\IEEEoverridecommandlockouts
% The preceding line is only needed to identify funding in the first footnote. If that is unneeded, please comment it out.
\usepackage{cite}
\usepackage{amsmath,amssymb,amsfonts}
\usepackage{algorithmic}
\usepackage{graphicx}
\usepackage{textcomp}
\usepackage{xcolor}
\usepackage{url}
% Times-compatible text and math fonts without missing TU/ptm shapes
\usepackage{newtxtext,newtxmath}
\def\BibTeX{{\rm B\kern-.05em{\sc i\kern-.025em b}\kern-.08em
    T\kern-.1667em\lower.7ex\hbox{E}\kern-.125emX}}

\begin{document}

\title{{Title}
    \thanks{Term Paper submitted for SC6101 Introduction to Blockchain, NTU.}
}

\author{
    \IEEEauthorblockN{Chen Sibei}
    \IEEEauthorblockA{G2509013K \\
        \textit{Graduate College}\\
        \textit{Nanyang Technological University}\\
        Singapore \\
        SCHEN050@e.ntu.edu.sg}\\
    \IEEEauthorblockN{Shan Xingzhong}
    \IEEEauthorblockA{Student's Matriculation Number \\
        \textit{Graduate College}\\
        \textit{Nanyang Technological University}\\
        Singapore \\
        Email address of Student}

    \and
    \IEEEauthorblockN{Feng Jiahao}
    \IEEEauthorblockA{Student's Matriculation Number \\
        \textit{Graduate College}\\
        \textit{Nanyang Technological University}\\
        Singapore \\
        Email address of Student}\\
    \IEEEauthorblockN{Tian Zhenman}
    \IEEEauthorblockA{Student's Matriculation Number \\
        \textit{Graduate College}\\
        \textit{Nanyang Technological University}\\
        Singapore \\
        Email address of Student}

    \and
    \IEEEauthorblockN{Mo Fan}
    \IEEEauthorblockA{Student's Matriculation Number \\
        \textit{Graduate College}\\
        \textit{Nanyang Technological University}\\
        Singapore \\
        Email address of Student}\\
    \IEEEauthorblockN{Zhang Boyu}
    \IEEEauthorblockA{G2509012B \\
        \textit{Graduate College}\\
        \textit{Nanyang Technological University}\\
        Singapore \\
        BZHANG037@e.ntu.edu.sg}
}

\maketitle

\begin{abstract}
    TODO: Abstract
\end{abstract}

\begin{IEEEkeywords}
    Stablecoins, Privacy-Preserving Techniques, KYC/AML Compliance, Regulatory Frameworks, Zero-Knowledge Proofs, Blockchain Privacy, Virtual Assets
\end{IEEEkeywords}

\section{Introduction}
TODO: Introduction

\section{On-Chain Traceability and Address Linkability}
Stablecoins on Ethereum and its compatible networks—primarily ERC-20 tokens such as USDC, USDT, and DAI—have become the cornerstone of the digital asset ecosystem. However, this convenience is built upon an extremely transparent architecture—the account-based model. Unlike Bitcoin's unspent transaction output (UTXO) model, the account-based model is designed to encourage address reuse and lacks a native change mechanism. This architectural characteristic, combined with the event logs mechanism of the ERC-20 standard, subjects stablecoin transactions to more severe privacy challenges than native cryptocurrencies.

\subsection{Threat Models}
When analyzing stablecoin privacy, it is essential to first define the scope and perspective of an attacker's capabilities. The academic community typically categorizes adversaries targeting blockchain privacy into three tiers: global passive adversaries, network-layer adversaries, and centralized issuers/regulatory adversaries.

\paragraph{Global Passive Adversary (GPA)}
GPA is assumed to possess a complete copy of the blockchain ledger (i.e., running an archival node), capable of parsing all block data, transaction records, and event logs generated by smart contracts since the genesis block.
\begin{itemize}
    \item \textbf{Capability:} The GPA constructs a complete transaction graph $G(V, E)$, where $V$ represents addresses and $E$ represents token or ETH transfers.
    \item \textbf{Goal:} To map multiple pseudonyms to a single real-world entity using clustering heuristics.
    \item \textbf{Reality:} This model corresponds to blockchain analytics firms (e.g., Chainalysis) and academic researchers. Victor (2020) demonstrated that GPAs can effectively cluster Ethereum users by exploiting deposit address reuse \cite{victor2020}.
\end{itemize}

\paragraph{Network-Level Adversary (NLA)}
Unlike GPA, which focuses solely on on-chain data, NLA monitors the underlying P2P network or RPC (Remote Procedure Call) communication traffic.
\begin{itemize}
    \item \textbf{Capability:} NLA attempts to associate transaction broadcast times or network packet characteristics with physical IP addresses.
    \item \textbf{Latest Threat:} Wang et al. (2025) proposed the “Time Tells All” attack model, demonstrating that NLA no longer requires operating supernodes or controlling large numbers of P2P connections. Instead, it can achieve de-anonymization simply by monitoring encrypted TCP traffic—such as through ISP or local network eavesdroppers. \cite{wang2025}.
\end{itemize}

\paragraph{Privileged Issuer Adversary}
This is a threat model specific to stablecoins, particularly fiat-collateralized stablecoins.
\begin{itemize}
    \item \textbf{Capability:} Issuers (such as Circle and Tether) not only possess an on-chain view of GPA but also control off-chain KYC (Know Your Customer) data. Furthermore, they hold god-like privileges at the smart contract level, including blacklisting and freezing addresses.
    \item \textbf{Impact:} This privilege enables taint analysis to be enforced. Once an address is flagged, other addresses that have interacted with it—even if the interaction occurred several hops prior—may face the risk of asset freezing.
\end{itemize}

\subsection{Attack and Inference Mechanisms}
The core of de-anonymization attacks lies in breaking the unlinkability between addresses. Under the Ethereum account model, academia has developed multiple high-precision heuristic algorithms. Combined with network-layer side-channel attacks, these techniques render stablecoin users' privacy virtually impossible to conceal.

\paragraph{Clustering Heuristics in the Account Model}
In Bitcoin, the Multi-input Heuristic serves as the gold standard for clustering. However, Ethereum transactions typically involve only a single sender, rendering Bitcoin's clustering methods ineffective. For Ethereum, researchers have proposed novel heuristic algorithms based on account behavior patterns.
\begin{itemize}
\item \textbf{Deposit Address Reuse:}
In his paper “Address Clustering Heuristics for Ethereum,” presented at the premier conference Financial Cryptography in 2020, Friedhelm Victor systematically quantified for the first time the privacy leakage resulting from address reuse.
\item \textbf{Gas Payer Linking (The ``Sugar Daddy'' Attack):}
Identified by B\'{e}res et al. (2020), this is a potent attack against ERC-20 privacy \cite{beres2020}. Since stablecoins cannot pay for their own execution gas directly, a fresh address ($Address_{New}$) receiving USDC needs ETH to move it. Users typically fund this gas from a main wallet ($Address_{Main}$) or a CEX.
$$ Address_{Main} \xrightarrow{\text{ETH}} Address_{New} $$
This funding transaction creates a visible link. Even if the stablecoins received by $Address_{New}$ are from a private source, the gas funding trail reveals the entity relationship, creating a ``star topology'' in the transaction graph.
\item \textbf{Behavioral Fingerprinting:}
Reuter (2025) highlights that time-zone analysis of transaction timestamps can geo-locate users \cite{reuter2025}. 
Furthermore, the set of contracts a user approves (via the \texttt{approve()} function) creates a unique fingerprint. 
If two addresses approve the same specific sequence of DeFi protocols (e.g., Uniswap V3, Aave, and a niche NFT market), they are likely controlled by the same user.
\end{itemize}

\paragraph{RPC/P2P Side-Channel Attacks}
Although on-chain clustering can only associate addresses with “entities,” network-layer attacks can link addresses to “IP addresses,” thereby enabling physical geolocation.
\begin{itemize}
\item \textbf{"Time Tells All" Attack:}
Wang et al. (2025) presented a de-anonymization attack targeting RPC users in a USENIX Security/CCS-level research paper. \cite{wang2025}
\item \textbf{"P2P Topology" Attacks:}
Early studies focused on message propagation delays at the P2P layer. 
While the impact is mitigated for users relying on RPC, further research by Heimbach et al. (2025) demonstrated that by analyzing proof messages (Attestations) within TCP connections, the physical locations of over 15\% of Ethereum validator nodes could be identified.\cite{heimbach2025}
\item \textbf{"Specific Leakage in ERC-20 Graphs":}
The transfer patterns of stablecoins exhibit significant topological differences from native ETH patterns, providing entry points for specific analyses.
Josenhans et al. (2025) noted in their analysis of ERC-20 scam token networks that the peel chain pattern—where large funds gradually diminish through successive transfers, akin to peeling an onion—is highly prevalent in token laundering. \cite{josenhans2025}
Using graph embedding technology, researchers achieved 88.7\% accuracy in identifying anomalous token transfer networks. 
For stablecoins, this pattern typically indicates users attempting to obscure fund paths via “bridge” addresses.
\end{itemize}

\subsection{Privacy Impact}
The combined application of the aforementioned attack mechanisms has resulted in stablecoin users facing a multidimensional collapse of privacy:

\paragraph{Physical Identity Exposure}
By combining CEX KYC data (obtained through deposit address reuse) with ISP IP data (obtained via RPC side-channels), attackers can fully map on-chain pseudonyms to real-world legal entities.

\paragraph{Wealth and Behavior Insights}
Reuter (2025) research demonstrates how on-chain behavioral analysis can map stablecoin flows to specific geographic regions. \cite{reuter2025}
When this macro-level insight falls into the hands of criminals, it may enable targeted phishing attacks or even offline extortion against high-value holders. 

\paragraph{Mining Efficiency Value (MEV) and Front-Running Transactions}
High-frequency traders leverage address clustering data to build “whale” profiles. 
When flagged stablecoin holders initiate transfers, MEV bots exploit the gas bidding mechanism to execute front-running or sandwich attacks, causing direct financial losses to users.

\subsection{Mitigations}
To address these threats, the academic community and developer community have proposed various cryptography-based mitigation solutions, primarily focusing on stealth addresses and privacy pool technologies.

\paragraph{Stealth Addresses}
Stealth addresses are designed to sever on-chain links between senders and recipients, preventing third parties from identifying who received the funds.
\begin{itemize}
    \item \textbf{Mechanism Principle:} 
    Based on the ERC-5564 standard and the Umbra protocol, stealth addresses utilize elliptic curve Diffie-Hellman (ECDH) key exchange. The process begins with the recipient publishing a “Meta-Address.” The sender uses this Meta-Address to generate a temporary, one-time Stealth Address and sends funds to it. Consequently, only the recipient can compute the private key for this Stealth Address and control the funds.   
    \item \textbf{Academic Assessment and Gas Defects:} 
    Kovács and Seres (2024) conducted a rigorous academic audit of Umbra in their paper “Anonymity Analysis of the Umbra Stealth Address Scheme on Ethereum.” 
    They identified a critical privacy vulnerability—the Gas Problem. \cite{kovacs2024}.
    After receiving USDC, the stealth address itself lacks ETH to pay the gas fees required to transfer these USDC. If the recipient transfers ETH from their main wallet to the stealth address, anonymity is instantly compromised (via Gas Payer Linking).
\end{itemize}

\paragraph{Privacy Pools}
Following the sanctions imposed on Tornado Cash, Vitalik Buterin, Jacob Illum et al. 
(2024) proposed the concept of “privacy pools” in an attempt to strike a balance between privacy and compliance\cite{buterin2024}.
\begin{itemize}
    \item \textbf{Mechanism Principle:} 
    Privacy pools no longer pursue “full-set anonymity,” but instead introduce the concept of Association Sets. 
    Users leverage zero-knowledge proofs (ZK-SNARKs) to prove:
    \textit{1) I hold a deposit in the pool.}
    \textit{2) My deposit originates from a specific “Proof of Inclusion” set.}
    \textit{3) Alternatively, my deposit origin does not belong to any known “Proof of Exclusion” set of hacked/sanctioned addresses.}
    \item \textbf{SeDe:} 
    This solution implements Selective De-anonymization (SeDe), enabling honest users to prove their innocence while severing ties with illicit funds.\cite{sahu2024sede}
    \item \textbf{Anonymous Collection Fragmentation Risk:} 
    Academic circles express concern that this mechanism may lead to fragmentation of anonymity sets. 
    If each user selects different association sets, the scale (k-anonymity) of each set will significantly decrease, thereby reducing overall privacy protection. 
    Furthermore, Buterin et al. (2024) also acknowledge that maintaining association sets introduces centralized or federated trust assumptions (who defines “good” and “bad”?), which could become a new point of entry for censorship \cite{buterin2024}.
\end{itemize}

\section{Privacy Leakage Through Collateral and Liquidation Mechanisms}

\section{Cross-Layer and Cross-Chain Correlation Attacks}

\section{Compliance-Driven Identity Disclosure}

Within digital asset markets, fiat-pegged stablecoins such as Tether (USDT) and USD Coin (USDC) have increasingly replaced direct fiat currency pairs as the dominant medium of exchange and settlement layer. Recent evidence suggests that stablecoin-denominated trading accounts for approximately 80\% of total cryptocurrency trading volume, underscoring their central role in the digital asset ecosystem \cite{waller2025,theblock2025}. This structural shift positions stablecoins as de facto on-ramps and off-ramps for digital assets, thereby amplifying concerns surrounding privacy erosion and compliance-driven control. In this section, we will use USDT as a case study, to examine the divergence between original design assumptions and current operational reality, followed by a discussion of proposed mitigation approaches.

\subsection{Problem: Privacy, KYC, and Regulatory Drift in USDT}

USDT was originally proposed as a blockchain-based representation of fiat currency, designed to facilitate seamless peer-to-peer transactions without intermediaries \cite{tether2014}. However, over time, Tether Limited has increasingly adopted centralized control mechanisms, including Know Your Customer (KYC) protocols and transaction monitoring systems, ostensibly to comply with evolving regulatory requirements. This shift has led to a divergence between the original vision of USDT as a privacy-preserving digital asset and its current operational reality as a regulated financial instrument.

While USDT transactions are recorded on public blockchains, Tether Limited maintains the ability to freeze or blacklist addresses associated with illicit activities, effectively undermining the pseudonymous nature of blockchain transactions. Furthermore, the implementation of KYC procedures for USDT issuance and redemption introduces additional vectors for privacy leakage, as user identities can be linked to on-chain activities. These compliance measures are driven by regulatory frameworks such as FinCEN's guidance on virtual currencies \cite{finCEN2019}, FATF's risk-based approach to virtual assets \cite{fatf2021}, and OFAC's sanctions compliance guidance \cite{ofac2021}.

Furthermore, despite originally deployed over the Bitcoin blockchain via the Omni Layer protocol, USDT has since expanded to multiple blockchains, including Ethereum, Tron, and Solana. Each of these platforms has distinct privacy characteristics and regulatory compliance mechanisms, further complicating the privacy landscape for USDT users. The multi-chain deployment fragmented liquidity and complicated regulatory oversight, increasing reliance on off-chain intermediaries for compliance enforcement rather than deterministic compliance guarantees \cite{micar2023}.

These factors collectively contribute to a complex privacy environment for USDT users, where the interplay between on-chain transparency and off-chain regulatory compliance creates significant challenges for maintaining user privacy. In the subsequent subsection, we will explore potential mitigation strategies and design trade-offs to address these challenges.

\subsection{Mitigation Approaches and Design Trade-offs}

Regulatory frameworks worldwide mandate KYC and AML compliance for financial service providers, including stablecoin issuers, making complete anonymity incompatible with legal operation in most jurisdictions. The challenge, therefore, is not to eliminate compliance requirements but to design systems that satisfy regulatory obligations while minimizing privacy intrusions. Existing literature proposes several approaches to achieve this balance.

One approach embeds compliance verification mechanisms at the protocol layer of stablecoin systems. Recent work by Duffie et al. \cite{duffie2025} proposes that stablecoin architectures can reconcile strong privacy protections with regulatory obligations (including sanctions screening, KYC, AML, and CFT requirements) by integrating privacy-preserving verification directly into the distributed ledger design, thereby avoiding the current reliance on centralized off-chain intermediaries.

A second approach leverages cryptographic techniques for selective disclosure, allowing users to prove compliance attributes without exposing underlying transaction data or identities. Baldimtsi et al. \cite{baldimtsi2024} systematize over a decade of privacy-preserving payment systems, examining how zero-knowledge proofs and related cryptographic methods enable different privacy guarantees ranging from confidentiality to k-anonymity and sender-receiver unlinkability. However, their analysis identifies a fundamental trilemma: existing systems cannot simultaneously achieve strong anonymity, scalable on-chain data growth, and efficient light-client verification. This limitation suggests that stablecoin designers must make explicit trade-offs between privacy strength, system scalability, and accessibility for resource-constrained users.

A third category employs tiered or hybrid KYC models, as outlined in regulatory frameworks such as FATF's risk-based guidance \cite{fatf2021}. Under this approach, Virtual Asset Service Providers (VASPs) may apply lighter customer due diligence (CDD) procedures for occasional transactions below specified thresholds, such as the USD/EUR 1,000 benchmark in FATF guidance, while requiring full identification and enhanced due diligence for higher-value or higher-risk activities. However, recognizing the unique risk profile of virtual assets, including rapid global settlement and potential for anonymity, some jurisdictions opt to mandate comprehensive KYC regardless of transaction size. This flexibility allows regulatory regimes to calibrate privacy-compliance trade-offs based on jurisdiction-specific risk assessments.

While no single approach fully resolves the fundamental privacy-regulation tension, these mechanisms suggest that future stablecoin architectures may achieve improved privacy without sacrificing regulatory compatibility, provided that design decisions appropriately balance competing policy objectives.

\section{Conclusion}
TODO: Conclusion

% References are listed in the file named "references.bib"
\bibliographystyle{IEEEtran}
\bibliography{references}

\end{document}
